\documentclass{article}

\usepackage[utf8]{inputenc}
\usepackage{tikz}
\usepackage{tkz-euclide}
\usepackage{bbm}
\usepackage{mathtools}
\usepackage{amssymb}
\usetikzlibrary{shapes.geometric}
\usetikzlibrary{calc,intersections,angles,quotes}
\title{Proof}
\author{Faris B. Mismar}
\date{September 2021}

\begin{document}

\maketitle

\section{Introduction}

Let an arbitrary star $\mathcal{S}$ have a length $\ell$ and $N$ points, with $N, \ell \in \mathbbm{Z}_{++}$.  To draw this star, as shown in g.~\ref{fig:star}, the angle that the graphics turtle would need to rotate in each of the $N$ line segments of length $\ell$ is:
\begin{equation}
    \theta = \frac{360^\circ}{N} \times 2.
\end{equation}%

Let $\alpha$ be the measure of the circumferential angles formed by the vertices of the star $\mathcal{S}$ at the circumference of the inscribing circle.  It is easy to show that the radius of the circle $r$ splits $\triangle AXY$ into two identical triangles and therefore $\angle XAO = \alpha/2$.  Due to the same reason, the interior angle of the $N$-polygon formed by the star $\angle XOY = \frac{\theta}{4}$.

It should be easy to find out that $\alpha = 180^\circ - \theta$ since both angles fall on a straight line.  These two angles are \textit{supplementary}.

We inspect $\triangle AOB$ and $\triangle ADB$ knowing that $\angle ADB \coloneqq \alpha$ due to symmetry.  Therefore, the central angle $\angle BOA$ sharing the same arc with the circumferential angle has the measure of $2\alpha$.

$\triangle AOB$ is an isosceles triangle with the base angle measure of $\gamma$.  Therefore, the measure of $\angle BOA$ is $\zeta \coloneqq 180^\circ - 2 \gamma$.  We can use the law of sines and write:

\begin{equation}
    \frac{r}{\sin \gamma}  = \frac{\overline{\rm AB}}{\sin \zeta},
\end{equation}
which makes it easy to write $r$ as
\begin{equation}
\begin{aligned}
r &= \overline{\rm AB} \frac{\sin \gamma}{\sin (180^\circ - 2\gamma)} \\
  &= \overline{\rm AB} \frac{\sin \gamma}{\sin 2\gamma} \\ 
  &= \overline{\rm AB} \frac{1}{2\cos \gamma}, \\ 
\end{aligned}
\end{equation}%
where the last step comes from the trigonometric identity of the sine of a double-angle.  We can write $\gamma$ in terms of the circumferential angle $\alpha$ since $\zeta = 2\alpha = 180^\circ - 2\gamma$ (all from $\triangle AOB$).  Thus:
\begin{equation}
    \gamma = \frac{1}{2} (180^\circ - 2\alpha) = 90^\circ - \alpha.
\end{equation}

Now we can write $r$ using the cosine of the complementary angle as follows
\begin{equation}
\label{eq:r_prelim}
r = \frac{\overline{\rm AB}} {2 \sin \alpha}. \\ 
\end{equation}

We inspect $\triangle ABD$, which again is an isosceles triangle.  The base angles have the measure of $\gamma + \alpha / 2 = 90^\circ - \alpha / 2$ each.  The law of sines enables us to write:

\begin{equation}
    \frac{\ell}{\sin(90^\circ - \alpha / 2)}  = \frac{\overline{\rm AB}}{\sin \alpha}.
\end{equation}

Here we can compute:

\begin{equation}
    \overline{\rm AB} =  \ell \frac{\sin \alpha}{\cos \alpha/2} \\
    \label{eq:r_prelim_2}
\end{equation}%
using the trigonometric identity of the sine of a double-angle again.  What is left now is finding $r$  in terms of $N$ and $\ell$, which is obtained by substituting $\overline{\rm AB}$ of \eqref{eq:r_prelim_2} into \eqref{eq:r_prelim}:

\begin{equation}
r = \frac{\ell}{2\cos \alpha / 2}  = \frac{\ell}{2\sin \frac{360^\circ}{N}}
\end{equation}

\begin{equation}
\therefore r  = \frac{\ell}{2} \csc \left ( \frac{360^\circ}{N} \right ) \qquad \blacksquare
\end{equation}

\begin{figure}[!t]
\centering
\input{triangle.tikz}
\caption{Solving the triangles}\label{fig:star}
\end{figure}

\end{document}
